\documentclass{article}

\usepackage[english]{babel}

\author{Florian Thuin}
\title{SINF2224 - Programming Methods : Exam questions}

\begin{document}
  \section{Question 1}
  Describe the differences and relations between \textit{semantic proofs} and
  \textit{axiomatic proofs}. Give an example of a technique belonging to each
  category. Discuss their respective merits. \newline

  \section{Question 2}
  Describe the general principles of \textit{Hoare logic} (syntax and
  semantics). Explain what an \textit{inference system} on this logic is, what
  that is used for and how that works. \newline

  \section{Question 3}
  Can one build a \textit{totally automatic} proof system for Hoare logic on a
  language like Java? Discuss difficulties that occur and approaches to address
  those difficulties. \newline

  \section{Question 4}
  Make the link between resolving \textit{program proofs} and resolving
  \textit{proofs in classical logic}. On that basis, explain the notion of
  \textit{relative completeness} of a program proof system. \newline

  \section{Question 5}
  Describe the principles of \textit{pre-condition calculus} and of generation
  of \textit{verification conditions}. Explain the practical usefulness of those
  computations. Make the link with Hoare logic and its inference rules. \newline

  \section{Question 6}
  Explain the notion of \textit{loop invariant} and its usefulness. Illustrate
  with a proof rule. In which way does this affect automatic program proofs?
  \newline

  \section{Question 7}
  Distinguish the notions of \textit{partial} and \textit{total correctness}.
  Discuss aspects of program proofs where those notions become relevant. Explain
  the notion of \textit{variant} and its usefulness, for loops and procedures.
  \newline

  \section{Question 8}
  Cite and explain the proof rule for \textit{(non-recursive) procedure calls}
  that corresponds to the approach used in ESC/Java. Discuss its pros and cons.
  What should be modified for recursive calls? \newline
  \section{Question 9}
  Based on a simple algebraic datatype (e.g. lists), explain what is a \textit{
  definition by structural induction} and discuss the advantages of such a
  definition when proving programs. \newline
  \section{Question 10}
  Based on a simple example, explain why Hoare logic proofs are \textit{not
  compositional} on \textit{concurrent programs}. Succintly discuss the
  particular cases of \textit{disjoint parallel programs} and \textit{
  interference-free programs}. \newline
  \section{Question 11}
  Describe how the notion of \textit{deadlock} occurs in program proofs. Define
  \textit{strong} vs. \textit{weak} total correctness. Distinguish the case
  of sequential programs (or components), then concurrent programs. \newline
  \section{Question 12}
  Show the principles of \textit{linear temporal logic} (LTL) and its
  interpretation over Kripke structures. Describe an example of LTL property.
  \newline
  \section{Question 13}
  Describe the general principle of \textit{model-checking} for simple
  properties (e.g. invariants). Compare with deductive proofs, in terms of
  limitation, automation, kinds of properties to be checked. \newline
  \section{Question 14}
  Describe the principle of \textit{model-checking for LTL} using Büchi
  automata. Discuss its algorithmic complexity. Mention a method that can
  improve the efficiency of model-checking. \newline
\end{document}
