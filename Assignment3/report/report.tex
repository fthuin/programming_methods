\documentclass[11pt, a4paper]{article}

\usepackage[english]{babel}
\usepackage[utf8]{inputenc}
\usepackage[T1]{fontenc}
\usepackage{csquotes}
\usepackage{listings}
\usepackage{booktabs}

\usepackage{color}

\definecolor{pblue}{rgb}{0.13,0.13,1}
\definecolor{pgreen}{rgb}{0,0.5,0}
\definecolor{pred}{rgb}{0.9,0,0}
\definecolor{pgrey}{rgb}{0.46,0.45,0.48}

\lstset{language=Java,
  showspaces=false,
  showtabs=false,
  breaklines=true,
  frame=single,
  showstringspaces=false,
  breakatwhitespace=true,
  commentstyle=\color{pgreen},
  keywordstyle=\color{pblue},
  morekeywords={assert},
  stringstyle=\color{pred},
  basicstyle=\ttfamily\footnotesize
}

\title{SINF2224 --- Assignment 3}
\author{Florian Thuin \and Syméon Malengreau}
\date{\today}

\begin{document}
    \maketitle

    \section{Analytical expression of the number of different test cases}

    The analyze of the function \verb#concurrentTransactions#$(N_t, N_a, M_i, M_t)$
    to find the number of states $N_s$ can be separated in two parts:

    \begin{itemize}
        \item The first \verb#while# loop can create $N_a$ accounts with a
        random value between 1 and $M_i$, such that:
            $$N_s = {M_i}^{N_a}$$
        \item The second \verb#while# loop can create from two random accounts
        (identified by their unique id from 1 to $N_a$) a random tranfer from 1
        to $M_t$
            $$N_s = {(N_a \times N_a \times M_t)}^{N_t}$$
    \end{itemize}

    The result for the whole function can be expressed as:

    $$N_s = {M_i}^{N_a} \times {(N_a \times N_a \times M_t)}^{N_t}$$

    \section{Experiments}

    Those experiments were made in Salle Intel, the computers are becoming really
    old and we can't remove all other processes, such that we had very poor results
    and we decided to use the modified test harness (see the next section).
    It just removes the non-relevant states so it won't affect the
    relation with the analytical expression. You can reproduce them by
    running \verb#test.sh#.

    \begin{tabular}{cccccc}
        \toprule
        $(N_t, N_a, M_i, M_t)$ & New & Visited & End & Time (mm:ss) & Memory \\
        \toprule
        (2,2,2,2)            & 15393 & 11031  & 1718 & 00:07 & 166MB \\
        \midrule
        (3,2,2,2)           & 397703 & 320454 & 12860 & 01:27 & 377MB \\
        \midrule
        (2,3,2,2)           & 254696 & 206144 & 26966 & 00:45 & 166MB \\
        (2,4,2,2)           & 2336533 & 2006172 & 236338 & 06:45 & 515MB \\
        \midrule
        (2,2,3,2)           & 33855 & 24429 & 3728 & 00:08 & 166MB \\
        (2,2,4,2)           & 59325 & 43007 & 6462 & 00:12 & 166MB \\
        (2,2,5,2)           & 91803 & 66765 & 9920 & 00:17 & 166MB \\
        (2,2,6,2)           & 131289 & 95703 & 14102 & 00:25 & 356MB \\
        (2,2,7,2)           & 177783 & 129821 & 19008 & 00:32 & 377MB \\
        (2,2,8,2)           & 231285 & 169119 & 24638 & 00:40 & 377MB \\
        (2,2,9,2)           & 291795 & 213597 & 30992 & 00:51 & 356MB \\
        (2,2,10,2)          & 359313 & 263255 & 38070 & 01:02 & 288MB \\
        \midrule
        (2,2,2,3)           & 29190 & 23024 & 3067 & 00:07 & 166MB \\
        (2,2,2,4)           & 46587 & 38805 & 4736 & 00:10 & 288MB \\
        (2,2,2,5)           & 68537 & 59060 & 6755 & 00:14 & 190MB \\
        (2,2,2,6)           & 94424 & 83270 & 9100 & 00:20 & 377MB \\
        (2,2,2,7)           & 124853 & 111944 & 11787 & 00:25 & 377MB \\
        (2,2,2,8)           & 159150 & 144498 & 14798 & 00:31 & 288MB \\
        (2,2,2,9)           & 198063 & 181590 & 18151 & 00:39 & 247MB \\
        (2,2,2,10)          & 240783 & 222509 & 21834 & 00:47 & 377MB \\
        \bottomrule
    \end{tabular}
    \bigskip

    So as we can see in this table, the exponent factors in the analytic
    formula really play an important role in the state-space explosion. We can
    conclude that, even if it is not a perfect formula (because many other
    factors come from what Java does internally and how JPF handles it),
    it globally reflects where will be the limits of the proofs.

    \section{Modified test harness}

    We can delete the printf inside the tests method to reduce the number of
    states. We can also delete all unused variables (that were needed by
    those printf). Only by doing that, with (2, 2, 2, 2) as parameters, the
    number of visited states decreases from 57835 to 11031 and the computation
    time is already divided by 5. \newline

    This is a very easy improvement, there might be many others directly
    related to the program (as not trying to transfer from an account that
    already has a balance of 0 to withdraw 0 and deposit 0) but we gave a try
    to them and it only increased the number of states.

    \section{Ensures balance never becomes negative}

    First, we have to be sure that the initialization never sets a negative
    balance by adding this line in the constructor:

    \begin{lstlisting}
assert 0 <= initialAmount;
    \end{lstlisting}

    We also have to add it to the methods that modifies the balance, such as
    \verb#withdraw# and \verb#deposit#:

    \begin{lstlisting}
private void withdraw(final int amount) {
    assert balance - amount >= 0;
    balance = balance - amount;
}
private void deposit(final int amount) {
    assert balance + amount >= 0;
    balance = balance + amount;
}
    \end{lstlisting}

    \section{Verifying merge method}

    To verify \verb#merge# method, we can create more threads in the
    \verb#concurrentTransactions# that will call it (concurrently from the ones
    that do the \verb#transferTo#), we don't really care about the number of
    merge (because it is not a parameter) so you can pay no attention to that.
    Here is what we can write:

    \begin{lstlisting}
i = 0;
while (i != nbrOfTransfer) {
    final Thread t = new Thread() {
        @Override
        public void run() {
            final Account a1 = accounts[random.nextInt(accounts.length)];
            final Account a2 = accounts[random.nextInt(accounts.length)];
            if (a1.id() != a2.id()) a1.merge(a2);
        }
    };
    t.start();
    i = i + 1;
}
    \end{lstlisting}

    But as JPF will output, we just created a deadlock\ldots So let's add some
    code to the merge method:

    \begin{lstlisting}
synchronized (getFirstSync(other)) {
    synchronized (getSecondSync(other)) {
        this.balance = this.balance + other.balance;
        other.balance = 0;
    }
}
    \end{lstlisting}

    Now everything works fine for this method.

    \section{Program errors found}

    In this section, we explain the way we corrected the program following the
    warning and errors given by running JPF on TestHarness. \newline

    The first program error was a \textbf{computation mistake} in \verb#transferTo#,
    such that balance could have been negative and the assertion would not
    be true when the amount to transfer is less than the account balance
    (JPF tells us it is a \verb#java.lang.AssertionError#):

    \begin{lstlisting}[language=Java]
withdraw(amount);
other.deposit(maxAmount);
    \end{lstlisting}

    has to become:
    \begin{lstlisting}[language=Java]
withdraw(maxAmount);
other.deposit(maxAmount);
    \end{lstlisting}

    The second program error was a \textbf{missing synchronization} on the access of the
    field \verb#balance# in the method \verb#balance()# which resulted in an
    \textit{unprotected field access}:

    \begin{scriptsize}
    \begin{verbatim}
[SEVERE] unprotected field access of: bank.Account@272.balance in thread: Thread-1 (bank/Account.java:49)
    \end{verbatim}
    \end{scriptsize}

    To correct that, we just have to synchronize it:

    \begin{lstlisting}[language=Java]
public int balance() {
    synchronized (this) {
        return balance;
    }
}
    \end{lstlisting}

    But correcting this error leads to create a deadlock (in fact, JPF now
    finds it) and there is still an assertion failure. \newline

    The third program error is that the \textbf{assertion fails} because the assignment
    of \verb#initialSum# is outside the synchronized part such that it can still
    change before the synchronisation. We have to move this line inside the scope
    of the synchronisation. \newline

    \begin{lstlisting}[language=Java]
final int initialSum = this.balance() + other.balance();
    \end{lstlisting}

    The fourth program error was a \textbf{synchronization} mistake, as we saw
    in the lab session with the dining philosophers problem, we have to lock
    and release the resources in a way that doesn't create deadlock (use the
    same order to lock and then release in the reverse order). Here, we can
    use the unique id of each account to lock first the account with the lower
    id and then the account with the greatest id. \newline

    \begin{lstlisting}[language=Java]
synchronized (getFirstSync(other)) {
    synchronized (getSecondSync(other)) {
        ...
    }
}
private Account getFirstSync(Account other) {
    if (this.id() < other.id()) return this;
        else return other;
}
private Account getSecondSync(Account other) {
    if (this.id() > other.id()) return this;
        else return other;
    }
    \end{lstlisting}

    \section{Conclusion}

    We didn't find JPF to be very very verbose because the errors returned were
    usually rather vague, but we tried our best to show that we understood how
    JPF works and that it helped us to submit a program that is errors-free.
    It was fun to try a program that is able to understand when two states
    of the program are equals and that can detect errors with threads.

\end{document}
